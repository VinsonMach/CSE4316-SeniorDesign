%This section provides a description of your product and defines it's primary features and functions. The purpose is to give the document reader/reviewer enough information about the product to allow them to easily follow the specification of requirements found in the remainder of the document. Your header for this section should introduce the section with a brief statement such as: "This section provides the reader with an overview of X. The primary operational aspects of the product, from the perspective of end users, maintainers and administrators, are defined here. The key features and functions found in the product, as well as critical user interactions and user interfaces are described in detail." Using words, and pictures or graphics where possible, specify the following:

This section provides a general overview of the Cerberus wrist module system. Described here are the features that our product provides, how our product interacts with its environment and user and how the interface will look for the Cerberus app and wrist module.
\subsection{Features \& Functions}
%What the product does and does not do. Specify in words what it looks like, referring to a conceptual diagram/graphic (Figure X).  Define the principle parts/components of the product. Specify the elements in the diagram/graphic that are part(s) of this product as well as any associated external elements (e.g., the Internet, an external web server, a GPS satellite, etc.)

The Cerberus system alerts wearers of events like their doorbell ringing, a door or window opening or noise around them. Figure 1 shows what happens when the doorbell ringing event occurs. The Cerberus system has hardware and software components. The wrist module notifies the wearer by vibrating on the wrist. The Cerberus app, which will be available for Android, identifies the event that caused the vibration via a notification. Users will create an account on the app to register their wrist module with their phone. In addition, the user can alter certain settings in the app to further control the wristband. The Cerberus server will store the user's account information like settings, username, password and phone number. The server will communicate externally with the Ring system, which the product will depend on for notifications. 

\subsection{External Inputs \& Outputs}
%Describe critical external data flows. What does your product require/expect to receive from end users or external systems (inputs), and what is expected to be created by your product for consumption by end users or external systems (outputs)? In other words, specify here all data/information to flow into and out of your systems. A table works best here, with rows for each critical data element, and columns for name, description and use.

\begin{table}[h]
\begin{tabular}{|l|l|l|}
\hline
Data Element & Input/Output & Description \\ \hline
Ring System notifications & Input & \begin{tabular}[c]{@{}l@{}}Notifications sent from the Ring System will be \\ intercepted by our server. These notifications will \\ be pushed to our phone app.\end{tabular} \\ \hline
App notifications & Output & \begin{tabular}[c]{@{}l@{}}The app will relay the notifications being sent \\ from the Ring system. The relays will come in\\ the form of notifications displayed to the user. \\ The relaying will also trigger the wrist module \\ vibration.\end{tabular} \\ \hline
Physical vibrations & Output & \begin{tabular}[c]{@{}l@{}}This will come from the wrist module. When \\ the Cerberus app receives a notification from \\ Ring, the wrist module will vibrate.\end{tabular} \\ \hline
\end{tabular}
\end{table}

\subsection{Product Interfaces}
%Specify what all operational (visible) interfaces look like to your end-user, administrator, maintainer, etc. Show sample/mocked-up screen shots, graphics of buttons, panels, etc. Refer to the critical external inputs and outputs described in the paragraph above.
Nothing to show yet.