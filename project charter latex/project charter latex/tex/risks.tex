One of the main tasks of project managers is to predict and solve risk factors that may affect the project schedule or software quality in advance. This is called "Risk Management." In fact, it is difficult to find a project that manages risks while looking forward from the beginning. Therefore, the projects usually feel the need for risk management far later, and at this time, many early errors have already occurred and settled. Our team wants to lay the groundwork for a successful project by predicting in advance the reasons for the occurrence and the time to be delayed so that team is prepared for the risks.

There are several types of risks, and representative examples include cases where experienced programmers quit the project, suspension of the project due to budget problems, or disagreement with clients. Considering the realistic situation of our team, we need to find out other risk factors because these are unlikely to happen, and the risk factors that have occurred with high probabilities so far are as follows.

\begin{table}[h]
\resizebox{\textwidth}{!}{
\begin{tabular}{|l|l|l|l|}
\hline
 \textbf{Risk description} & \textbf{Probability} & \textbf{Loss (days)} & \textbf{Exposure (days)} \\ \hline
 Campus lab space temporarily close due to Covid or other reasons  & 0.25 & 60 & 15.0 \\ \hline
 Delays in team meeting due to busy school schedules  & 0.40 & 14 & 5.6 \\ \hline
 Emergent idea meetings with the sponsor due to technical limitations  & 0.35 & 12 & 4.2 \\ \hline
 Delays in getting permission due to over budget & 0.25 & 12 & 3.0 \\ \hline
 Delays in shipping from out of state  & 0.10 & 15 & 1.5 \\ \hline
\end{tabular}}
\caption{Overview of highest exposure project risks} 
\end{table}