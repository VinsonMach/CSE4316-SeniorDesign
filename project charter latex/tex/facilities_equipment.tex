%What lab space, testing grounds, makerspaces, etc. will you need to complete the project? Will you require any specific equipment, and if so, where will you get it (borrow, lease, purchase, outsource, already present in the lab, etc.). This section should occupy 1/2 page.
Our team will be utilizing the lab space J, in room 208 that has been assigned to us. We may be utilizing the makerspace in the EE lab, which is located in 121 Nedderman hall. This is because we may want to create our own PCB for our wrist module. Currently our plan is to create our own wrist module, so we will need to purchase parts for that if we go that route. Parts will include an Arduino Nano, audio sensors and small motors. If we can not successfully make our own wrist module, we will utilize a prefabricated one like a fitbit. Our sponsor Dr. Conly has four fitbit charge 4's but that model of fitbut does not have a microphone so we would not be able to complete all the functionality we wanted to implement. If we decide to use them, we will borrow them from the department. Also if we want to add the feature of the wrist module alerting the wearer if a door or window is opened, we will need a security system to test the functionality on. If we are able to find a security system with a open source API, we will acquire a prefabricated system. But if we run into trouble finding an available, open source security system we will have to make our own with an Arduino and sensors, which will mean more parts to purchase. Xavier Holliday, Matthew Blount, and Katie Baumann have a few Arduino kits and electrical components between them, so we will utilize as much as we can from our preexisting supply.
